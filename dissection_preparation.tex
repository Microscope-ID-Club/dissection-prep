\documentclass[]{article}
\usepackage{caption}
\usepackage{graphicx}
\graphicspath{ {./images/} }
\usepackage{fancyhdr}
%opening
\title{Preparing Moth Specimens for Dissection}
\author{Dr. Paul J. Palmer}



\pagestyle{fancy}
\makeatletter
\let\runauthor\@author
\let\runtitle\@title
\makeatother
\chead{\runtitle}
\lfoot{\runauthor}
\rfoot{\today}





\begin{document}

%\maketitle
\pagenumbering{gobble}
%\begin{abstract}
%
%\end{abstract}

%\section*{Pitfall traps for spider recording}

%\begin{center}
%	\centering
%	\includegraphics[width=0.3\linewidth]{images/pitfall}\hfill
%	\includegraphics[width=.3\textwidth]{images/alcohol}\hfill
%	\includegraphics[width=.3\textwidth]{images/complete}\hfill
%	\captionof{figure}{Placing the pitfall trap: Place the pitfall cup in a carefully dug hole; Fill the collector cup with 30 mL of food grade  \textit{Mono~Propylene~Glycol} and  place the inside the pitfall; Cover with the rain shield leaving a vertical gap of about 10~mm. }
%	
%\end{center}

These instructions explain how to prepare moth specimens for confirmation of  identification (ID) by microscopic examination. These are my own preferences, but other workers will probably accept specimens prepared in this way. ID often requires examination of both external and internal features so the whole specimen should be preserved in a near perfect condition as possible. Dissection of the genitalia requires removal of the abdomen and separation of the reproductive organs and their arrangement on a microscope slide. This is only possible if the specimen is completely dessicated since the presence of any moisture will result in decay and the growth of moulds obscuring the features required for ID. Traditionally specimens were presented set on pins in a perfectly dehydrated state ready for examination. 


% TODO: \usepackage{graphicx} required
%\begin{center}
%	\centering
%	\includegraphics[width=0.3\linewidth]{images/Rainshield-1}\hfill
%	\includegraphics[width=.3\textwidth]{images/Rainshield-2}\hfill
%	\includegraphics[width=.3\textwidth]{images/Rainshield-3}\hfill
%	\captionof{figure}{Assembling the rain shield using a tent peg and a PETG square  \newline  Note: protective film has been left in place for clarity.}
%	
%\end{center}




\end{document}
